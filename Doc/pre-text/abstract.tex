\pagenumbering{roman}
\chapter*{Resumen}
    \addcontentsline{toc}{chapter}{Resumen}
    La computación cuántica es una tecnología en auge que utiliza las propiedades de la mecánica cuántica de modo que permite reducir la complejidad en operaciones costosas computacionalmente gracias al paralelismo cuántico. En este proyecto se busca optimizar la clasificación de tumores cerebrales en imágenes médicas mediante un enfoque híbrido cuántico-clásico. Utilizaremos \textit{Quantum Feature Embedding}, específicamente \textit{Amplitude Encoding} para extraer representaciones eficientes de imágenes de resonancia magnética (MRI por sus siglas en inglés) y una red neuronal clásica para dar interpretación al resultado de dichos embeddings y realizar la clasificación, esto usando el conjuto de datos proporcionado por Muhammad Al-Zafar Khan et al. \cite{khan2024brain} con poco más de 3000 MRI, clasificadas entre meningiona, glioma y tumor pituitario.
    
    \textbf{Palabras claves:} Circuitos Cuánticos Variacionales, Codificación por Amplitud, Imágenes de Resonancia Magnética, Redes Neuronales Convolucionales, Tumores Cerebrales.
\newpage